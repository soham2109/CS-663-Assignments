\documentclass{article}
\usepackage[utf8]{inputenc}

\title{Question 3}
\author{ }
\date{August 2020}

%\everymath{\displaystyle}


\begin{document}

\maketitle

\section{Problem Statement }

Consider a (non-discrete) image I(x) with a continuous domain and real-
valued intensities within [0, 1]. Let the image histogram be h(I), with mass
1. Consider the histogram h(I) is split into two histograms (i) $h_1 (I)$ over the
domain [0, a] and (ii) $h_2 (I)$ over the domain (a, 1], for some arbitrary $a \in (0,
1)$. Assume that the histogram mass within [0, a] is $\alpha \in (0, 1)$

\subsection{(8 points)}
\textbf{(a)} Suppose you perform histogram equalization over the two intensity in-
tervals [0, a] and (a, 1] separately, in a way that preserved the masses of the
two histograms $h_1 (I)$ and $h_2 (I)$ after the transformation. Derive the mean
intensity for the resulting histogram (or, equivalently, image) and include it
in the report.

\textbf{Given-}
\begin{equation}
\int\limits_0^1 h(I) \ dI =1 
\end{equation}
 
\begin{equation} \int\limits_0^a h_1(I) \ dI =\alpha = \int\limits_0^a h'_1(I) \ dI \end{equation} 
\begin{equation} \int\limits_a^1 h_2(I) \ dI =1-\alpha = \int\limits_a^1 h'_2(I) \ dI \end{equation} 

where $ \in (0, 1)$. In (1), (2) and, (3), h(I) represents the image
histogram of the image I(x). h(I) is split into two histograms at an inten-
sity $I_1 = a$ resulting in two new image histograms $h_1 (I)$ and $h_2 (I)$ over the
domains [0, a] and (a, 1] respectively. $h'_1 (I)$ and $h'_2$  (I) are the transformed
histograms after performing histogram equalization.

\textbf{To find-} Mean intensity of the resulting histogram after transformation
\begin{equation}E'[I] =  \int\limits_0^1 I h'(I) \ dI = \int\limits_0^a I h'_1(I) \ dI + \int\limits_a^1 I h'_2(I) \ dI = ?\end{equation}

\textbf{Solution-}

Let the histogram values after equalization be $h'_1$ and $h'_2$ respectively. Since after equalization, these values can be assumed to be constant,
Using (2) and (3)
\begin{equation} \int\limits_0^a h'_1(I) \ dI = h'_1 * I|_0^a = h'_1 * a = \alpha\end{equation}
\begin{equation} \int\limits_a^1 h'_2(I) \ dI = h'_2 * I|_a^1 = h'_2 *(1-a) = 1-\alpha\end{equation}
(5) and (6) give us relation between $h'_1$ , $h'_2$ , a and, $\alpha$. Using this relation
with (4) we get:
\begin{equation} \int\limits_0^a I h'_1(I) \ dI =\frac{h'_1 * I^2|_0^a}{2}= \frac{\alpha*\alpha}{2}\end{equation}
\begin{equation} \int\limits_a^1 I h'_2(I) \ dI =\frac{h'_2 * I^2|_a^1}{2}=\frac{\frac{1-\alpha}{1-\alpha}*(1-a^2)}{2}= \frac{(1-\alpha)*(1-\alpha)}{2}\end{equation}

Substituting values from (7) and (8) in (4) we get:

\begin{equation}-E'[I] = \frac{\alpha*\alpha}{20}+\frac{(1-\alpha)*(1-\alpha)}{2}=\frac{a\alpha+1+a-\alpha-a\alpha}{2}=\frac{1+a-\alpha}{2}\end{equation}
\subsection{(2 points)}
\textbf{(b)} Let the chosen intensity a be the median intensity for the original his-
togram h(I). Assume that the mean intensity for the original histogram h(I)
is also a. Then, what is the mean intensity for the resulting histogram (or,
equivalently, image). Show the derivations clearly in the report.

\textbf{Given-}
\begin{equation}\int\limits_0^a h(I) \ dI =\int\limits_a^1 h(I) \ dI = \frac{1}{2} \rightarrow \alpha = \frac{1}{2}\end{equation}
\begin{equation}E[I] = \int\limits_0^1 I h(I) \ dI =a\end{equation}

\textbf{To find-}
\begin{equation}E'[I] = \int\limits_0^1 I h'(I) \ dI = \int\limits_0^a I h'_1(I) \ dI + \int\limits_a^1 Ih'_2(I) \ dI = ?\end{equation}

\textbf{Solution-}
\begin{equation}E'[I] = \frac{1+a-\frac{1}{2}}{2} = \frac{0.5+a}{2}\end{equation}

\subsection{(5 points)}
\textbf{(c)} Describe a scenario where the above described histogram-based intensity
transform with a = $I_Median$ will do a better job in intensity transformation
than a simple histogram equalization. Explain the reasons clearly.

\textbf{Solution-}
We apply the standard histogram equalization algorithm on the image
histogram h(I) of the image I(x) with the following property:
\begin{equation}\int\limits_0^1 h(I)\ dI = 1\end{equation}
where the initial mean $E[I]$ is also taken to be as a. 

Let the value of the transformed histogram $h’(I)$ after the standard equalization be $h’$ which is a constant.
Then using conservation of mass,

\begin{equation}\int\limits_0^1 h(I)\ dI = \int\limits_0^1 h'(I)\ dI = h'*I|_0^1 = 1 \rightarrow h' = 1\end{equation}

Also,
\begin{equation}E'[I] = \int\limits_0^1 I h'(I)\ dI = \int\limits_0^1 I*h'\ dI =\frac{I^2|_0^1}{2}=\frac{1}{2}\end{equation}

The above result is independent of the components of the image. No mat-
ter what the content of the image is, the mean after equalization is always
going to be equal to $\frac{1}{2}$ .

On the other hand if we perform the histogram equalization using the
conditions given in \textbf{(b)} the results differ from (16). In this case the mean
$E'[I]$ can be written as:

\begin{equation}E'[I] = \frac{1+a-\alpha}{2}\end{equation}

We want to minimize the distance between the initial mean a and final
mean $E'[I]$ for which:

\begin{equation}min\quad d(a,E'[I]) = min\quad|a-E'[I]| = min_{\alpha\in(0,1)} \quad|a-\frac{1+a-\alpha}{2}|\end{equation}

which on optimizing results in $\alpha = \frac{1}{2}$ and thus concluding that a should
be the median of the image histogram $h(I)$,

\begin{equation}\int\limits_0^a h(I)\ dI = \alpha = \frac{1}{2}\end{equation}

which takes the mean of the finally transformed image histogram closer
to the initial mean a. In this case the mean value $E''[I]$ can be given as:

\begin{equation}E''[I] = \frac{0.5+a}{2}\end{equation}

This might be a good approach if we wish to keep the image’s overall
brightness intact.

\subsection{(10 points)}
\textbf{(d)} Do an online search to find an image along the lines of your reasoning.
Write a code for this intensity transformation and demonstrate the better
performance on the image you obtained. Note that the better performance
should be distinctly evident.

\textbf{Solution-}
\end{document}

